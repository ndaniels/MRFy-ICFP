
\documentclass[preprint]{sigplanconf}

% The following \documentclass options may be useful:
%
% 10pt          To set in 10-point type instead of 9-point.
% 11pt          To set in 11-point type instead of 9-point.
% authoryear    To obtain author/year citation style instead of numeric.

\usepackage{amsmath}

\begin{document}

\conferenceinfo{ICFP '12}{date, City.} 
\copyrightyear{2012} 
\copyrightdata{[to be supplied]} 

\titlebanner{banner above paper title}        % These are ignored unless
\preprintfooter{short description of paper}   % 'preprint' option specified.

\title{Experience Report: Haskell in Computational Biology}
\subtitle{Subtitle Text, if any}

\authorinfo{Noah M. Daniels \and Andrew Gallant \and Norman Ramsey}
           {Department of Computer Science, Tufts University}
           {\{ndaniels, agallant, nr\}@cs.tufts.edu}


\maketitle

\begin{abstract}
This is the text of the abstract.
\end{abstract}

\category{CR-number}{subcategory}{third-level}

\terms
term1, term2

\keywords
keyword1, keyword2

\section{Introduction}

We address this paper to computational biologists who may have an interest in 
functional programming, but this is not an advocacy paper. We do not argue that 
computational biologists \textit{should} use functional programming. Rather, 
argue that computational biologists \textit{can} successfully develop software 
in Haskell, and what to expect if they do.

We also address this paper to the functional programming community, in the 
hopes that our experiences will be enlightening. We believe that our 
experiences can provide lessons about outreach and education.


\subsection{Problem Domain}

Computational biologists write software to solve problems on large genomic or 
proteomic data sets. Traditionally, these software programs fall into two 
categories: those with high-performance computing needs are typically 
implemented in C++, while the remainder are typically implemented in Perl, 
Python, Ruby, or R. Recently, we have developed a program in Haskell which 
implements a novel algorithm for solving the remote homology detection problem. 
During this process, we have wondered: would the algorithm be easily expressed 
in Haskell? Would we be able to achieve acceptable runtime performance? How 
would our experience compare with our past experiences developing and 
maintaining computational biology software in C++, Ruby, and Python? What would 
be our greatest challenges? In this paper, we briefly explain the remote 
homology detection problem and our algorithm, and then attempt to answer the 
above questions.

As a brief refresher in the standard dogma of genetics, recall that genes 
(strands of deoxyribonucleic acid (DNA), which are sequences of the nucleotides 
adenosine, cytosine, guanine, and thymine, represented by the letters A, C, G, 
and T) are transcribed into ribonucleic acid (RNA). Some RNA is then translated 
into the 20 naturally-occuring amino acids, which form peptide chains -- 
proteins -- which fold into complex structures in the cell. These proteins are 
cellular machinery, performing the varied functions an organism needs to 
survive. Since the dawn of the human genome project in 1990, the cost of 
sequencing an entire genome has dropped exponentially. Recently, the doubling 
time for genomic data has been one year, in contrast to the eighteen month 
doubling time of Moore's law. However, determining what newly sequenced genes 
actually do -- when they become proteins, what structures those proteins fold 
into, and what functions they perform -- has not kept pace. Determining the 
atomic coordinates of a newly discovered protein may require as much as two 
graduate-student-years in a lab!

The problem of taking a newly found protein sequence about which only the 
sequence is known, and determining what existing proteins of known structure 
and function most closely resemble that new protein, is known as homology 
detection. When proteins of similar sequence can be found, this problem is 
largely solved. To solve the homology detection problem, computational 
biologists have developed fast, approximate methods for determining the 
structure and function of proteins based on their sequence. The most popular 
software for homology detection is called HMMER, which uses a hidden Markov 
model to capture evolutionary change.

However, when there are no known proteins of similar sequence, only 
evolutionarily distant proteins that may share similar structure, this problem 
is called remote homology detection. Hidden Markov model approaches begin to 
fail as sequence similarity falls off. Recently, Markov random field 
approaches, which capture non-local interactions in the protein structure, have 
shown to perform well. However, these approaches face the challenge of 
increased computational complexity; the SMURF program exhibits exponential time 
complexity, while SMURFLite (work by one of the authors) bounds the exponent by 
simplifying the Markov random field's dependency graph. We have implemented a 
stochastic search approach to Markov random fields, called MRFy.

The algorithm underlying MRFy grew out of an understanding of the limitations 
of the SMURF (Structural Motifs Using Random Fields) algorithm, which used 
multidimensional dynamic programming to compute the optimal parse of a query 
sequence onto a Markov random field.


TODO: (add explanation of MRFy architecture and algorithm)

\subsection{Our Background}

All three authors are members of the Tufts University Department of Computer 
Science. Noah Daniels has taken a graduate seminar in advanced functional 
programming, which included some Haskell, and spent ten years in industry as a 
professional programmer in languages such as Ruby, C, and C++. Andrew Gallant 
has taken a programming language heavily focused on functional programming, but 
with no Haskell. Norman Ramsey is the local expert, but he wrote no code for 
this project.


\section{What we did}

\subsection{Beautiful code}
MRFy contains two distinct algorithms: the classical Viterbi dynamic 
programming algorithm for hidden Markov models, and a stochastic search 
algorithm for placing beta strand positions. This stochastic search algorithm 
implements several search heuristics: random-mutation hill climbing, simulated 
annealing, multistart simulated annealing, and a genetic algorithm.


The Viterbi algorithm~\cite{Viterbi:1967} comprises a set of recurrence 
relations that seek to maximize the probability of an observed sequence (in 
this case, of amino acid ``letters'') begin emitted by a particular 
finite-state automaton (the hidden Markov model). The particular states of this 
finite-state automaton map to the recurrence relations, and in this instance 
derive from what is known in the computational biology community as a 
\textit{profile} hidden Markov model. The Viterbi algorithm in this instance 
recurses on ``match,'' ``insertion,'' and ``deletion'' states. To be efficient, 
the Viterbi algorithm relies on dynamic programming, or memoization, and the 
resulting asymptotic complexity is $O(|M|\times|N|)$, when $M$ is the sequence 
of distinct states (nodes in the automaton's graph) and $N$ is the sequence of 
letters in the protein sequence. In our particular problem domain, both $M$ and 
$N$ routinely contain several hundred to a few thousand elements.


We implemented the Viterbi algorithm in two distinct ways. We first wrote a 
bottom-up dynamic programming version, which began with the first node of the 
hidden Markov model and first observation and built up the three-dimensional 
array of possible paths through the model. The array containing the paths 
served as the memoization table. We next wrote a top-down dynamic programming 
version, which began with the last node of the model and the last observation, 
and applied structural recursion down to the base cases involving the first 
node or first observation (or both). In this second approach, we used the 
Data.Memocombinators library to provide the memoization container (again, a 
three-dimensional array). We determined that the run-time performances of both 
approaches were virtually indistinguishable, and we also noted that the 
top-down version quite faithfully resembled the recurrence relations found in a 
textbook~\cite{durbin} or the HMMER literature~\cite{eddy}. We chose to keep 
the top-down version, which later proved to be wise when we had to track down a 
bug in one of our base cases. This bug was observed as a missing state in the 
alignment our program displayed, and we quickly discovered that one of our base 
cases was missing this state: it was returning an empty list as the base of the 
path, rather than a list containing a single node. In this, we were grateful 
for the resemblance between the mathematical description of the algorithm and 
the top-down dynamic programming approach in Haskell, which resulted in 
perspicuous code.

breadcrumb: HOFs for search.







\subsection{Ugly code}




\appendix
\section{Appendix Title}

This is the text of the appendix, if you need one.

\acks

Acknowledgments, if needed.

% We recommend abbrvnat bibliography style.

\bibliographystyle{abbrvnat}

% The bibliography should be embedded for final submission.

\begin{thebibliography}{}
\softraggedright

\bibitem[Smith et~al.(2009)Smith, Jones]{smith02}
P. Q. Smith, and X. Y. Jones. ...reference text...

\end{thebibliography}

\end{document}
